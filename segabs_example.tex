\documentclass[wordlite,twoside,texlive,final,calfont]{segabs}
\usepackage{lipsum}
%\hypersetup{
%}
% An example of defining macros
\newcommand{\rs}[1]{\mathstrut\mbox{\scriptsize\rm #1}}
\newcommand{\rr}[1]{\mbox{\rm #1}}

\renewcommand{\figdir}{Fig}  % figure directory

\title{An example SEG expanded nonostract}

\renewcommand{\thefootnote}{\fnsymbol{footnote}} 

\author{Joe Dellinger\footnotemark[1], BP 
  and Sergey Fomel, University of Texas at Austin\footnotemark[2] and Joe DellingerII\footnotemark[1], BP 
  and Sergey Fomel, University of Texas at Austin}

\footer{Example}

% WHEN "twoside" is not passed to options, all pages will be configured as odd pages.
% left/right heads will be preferred shown on the even pages WHEN "twoside" appears in options.
% left/right heaqds will appear on odd pages WHEN \header is empty.
\lefthead{Dellinger \& Fomel}
\righthead{University of Texas at Austin}
% central head (\header) will be preferred shown on the odd pages WHEN "twoside" appears in options.
% central head will appear on even pages WHEN both \lefthead and \righthead are empty.
\header{SEG abstract example}

\begin{document}

\maketitle

\begin{abstract}
  This is an example of using the extended version of \textsf{segabs.cls} for writing
  SEG expanded abstracts. This example requires \texttt{extra} option, it has been enabled in default. Users could try to remove \texttt{wordlite} or \texttt{final}, or add \texttt{nohyref} to see what happens.
\end{abstract}

\section{Introduction}

This is an introduction. \LaTeX\ is a powerful document typesetting
system \cite[]{lamport}. An excellent reference is \cite[]{kopka}. The
new \textsf{segabs.cls} class complies with the \LaTeX2e\ standard.

\section*{Theory}

This is another section. 

\subsection{Equations}

Section headings should be capitalized. Subsection headings should
only have the first letter of the first word capitalized.

Here are examples of equations involving vectors and tensors:
\begin{equation}
\tensor{R} = 
\begin{pmatrix}
  R_{\rs{XX}} & R_{\rs{YX}} \cr R_{\rs{XY}} & R_{\rs{YY}}
\end{pmatrix}
=
\tensor{P}_{M\rightarrow R} \; \tensor{D} \; \tensor{P}_{S\rightarrow M}
\;\;\; \tensor{S} \ \ \  ,
\label{SVD}
\end{equation}
and
\begin{equation}
R_{j,m}(\omega) =
\sum_{n=1}^{N} \, \,
P_{j}^{(n)}(\mathbf{x}_R) \, \,
D^{(n)}(\omega) \, \,
P_{m}^{(n)}(\mathbf{x}_S) \ \ \ .
\label{SVDray}
\end{equation}

Note that the macros for the \verb#\tensor# command has been changed
to force tensors to be bold uppercase, in compliance with current SEG
submission standards. This is so that documents typeset to the old
standards will print out according to the new ones: e.g., tensor
$\tensor{t}$ (note converted to uppercase).

Let's check the \texttt{$\backslash$mathcal} command:
\begin{eqnarray}
  \mathcal{L} = \mathcal{D}(\Theta) + \mathcal{E}(\sigma).
\end{eqnarray}

\subsection*{Figures}

\autoref{fig:net_whole} shows what it is about.

\plot{net_whole}{width=\columnwidth}
{This figure is specified in the document by \texttt{
    $\backslash$plot\{net\_whole\}\{width=$\backslash$columnwidth\}\{This caption.\}}.
}

\subsubsection{Multiplot}

Sometimes it is convenient to put two or
more figures from different files in an array (see
\autoref{fig:block_incep,block_incepdec}). Individual plots are
\autoref{fig:block_incep} and~\autoref{fig:block_incepdec}.

\multiplot{2}{block_incep,block_incepdec}{width=0.8\columnwidth}
{This figure is specified in the document by \texttt{
    $\backslash$multiplot\{2\}\{block\_incep,block\_incepdec\}\\\{width=0.8$\backslash$columnwidth\}\{This caption.\}}.
}

The first argument of the \texttt{multiplot} command specifies the
number of plots per row.

\subsection{Tables}

The discussion is summarized in \autoref{tbl:example}.

\tabl{example}{This table is specified in the document by \texttt{
    $\backslash$tabl\{example\}\{This caption.\}\{\ldots\}}.
}{
  \begin{center}
    \begin{tabular}{|c|c|c|}
      \hline
      \multicolumn{3}{|c|}{Table Example} \\
      \hline
      migration\rule[-2ex]{0ex}{5ex} & 
      $\omega \rightarrow k_z$ & 
      $k_y^2+k-z^2\cos^2 \psi=4\omega^2/v^2$ \\
      \hline
      \parbox{0.55in}{zero-offset\\diffraction}\rule[-4ex]{0ex}{8ex} &
      $k_z\rightarrow\omega_0$ &
      $k_y^2+k_z^2=4\omega_0^2/v^2$ \\
      \hline
      DMO+NMO\rule[-2ex]{0in}{5ex} & $\omega\rightarrow\omega_0$ & 
      $\frac{1}{4}
      v^2k_y^2\sin^2\psi+\omega_0^2\cos^2\psi=\omega^2$ \\
      \hline
      radial DMO\rule[-2ex]{0in}{5ex} & $\omega\rightarrow\omega_s$ &
      $\frac{1}{4}v^2k_y^2\sin^2\psi+\omega_s^2=\omega^2$\\
      \hline
      radial NMO\rule[-2ex]{0in}{5ex} & $\omega_s\rightarrow\omega_0$ &
      $\omega_0\cos\psi=\omega_s$\\
      \hline
    \end{tabular}
  \end{center}
}

\subsection{Algorithms}

We show an example of algorithm in \autoref{alg:relu}. Users could use some commands like \texttt{$\backslash$STATE}, \texttt{$\backslash$FOR}, \texttt{$\backslash$FORALL}, \texttt{$\backslash$IF} and \texttt{$\backslash$WHILE} to write algorithms.

\begin{algorithm}[htbp]
  \caption{ReLU function.}
  \label{alg:relu}
  \begin{algorithmic}[1]
    \REQUIRE Input vector $\mathbf{x}$
    \ENSURE Output vector $\mathbf{y}$.
    % if-then-else
    \STATE $\mathbf{y} = \max(\mathbf{x}, 0)$;
  \end{algorithmic}
\end{algorithm}

\section{ACKNOWLEDGMENTS}

I wish to thank Ivan P\v{s}en\v{c}\'{\i}k and Fr\'ed\'eric Billette
for having names with non-English letters in them.  I wish to thank
\cite{Cerveny} for providing an example of how to make a bib file that
includes an author whose name begins with a non-English character and
\cite{forgues96} for providing both an example of referencing a Ph.D.
thesis and yet more non-English characters.

\append{Appendix example}

According to the new SEG standard, appendices come before references.
\begin{equation}
\frac{\partial U}{\partial z} = 
\left\{
  \sqrt{\frac{1}{v^2} - \left[\frac{\partial t}{\partial g}\right]^2} +
  \sqrt{\frac{1}{v^2} - \left[\frac{\partial t}{\partial s}\right]^2}
\right\}
\frac{\partial U}{\partial t}
\label{eqn:partial}
\end{equation}
It is important to get equation~\ref{eqn:partial} right.

\append{Another appendix}

\begin{equation}
\frac{\partial U}{\partial z} = 
\left\{
  \sqrt{\frac{1}{v^2} - \left[\frac{\partial t}{\partial g}\right]^2} +
  \sqrt{\frac{1}{v^2} - \left[\frac{\partial t}{\partial s}\right]^2}
\right\}
\frac{\partial U}{\partial t}
\label{eqn:partial2}
\end{equation}
Too lazy to type a different equation but note the numeration.

\plot*[!tb]{ops_seq2patch}{width=0.8\textwidth} 
{This figure is specified in the
  document by \texttt{
    $\backslash$plot*\{ops\_seq2patch\}\{width=0.8$\backslash$textwidth\}\{This caption.\}}.  }

The error comparison is provided in \autoref{fig:ops_seq2patch}.
  
\lipsum[1-2]

\onecolumn

%\append{The source of the bibliography}
%\verbatiminput{bib/example.bib}

\twocolumn

\bibliographystyle{seg}  % style file is seg.bst
\autobibliography{bib/example}

\end{document}
