% Using the Revised NRSM template which meets the standard of NRSM & IEEE totally.
\documentclass[]{NRSMRev}

\NRSMsetup{
  stitle       = {Template for AP-S/USNC-URSI Two-Page Summary (Replace these lines with your title.)},
  sauthor      = {Authors Name/s},
  subject      = {USNC-URSI National Radio Science Meeting},
  codeStyle    = {box}
}

% paper title
% can use linebreaks \\ within to get better formatting as desired
% If not configuredl, will use stitle in the setup directly.
\title{Template for AP-S/USNC-URSI Two-Page Summary (Replace these lines with your title.)}

% authors and affiliations
% the organization option [orgN] associates the authors with the
% proper address
\author[org1]{First A. Author}
\author[org1]{Second B. Author}
\author[org2]{Third C. Author}

% each address must have a unique identifier in the option field
\address[org1]{The University of Here, Anywhere, VA 12345, USA, http://www.here.edu}
\address[org2]{The Next Company, Neverland, WA 54321, USA (author@next.com)}

\begin{document}

% make the title area
\maketitle
% Now equivalent to \newmaketitle

\begin{abstract}
  Please include a brief abstract here. The abstract should be limited to 50-200 words and should concisely state what was done, how it was done, principal results, and their significance.
\end{abstract}



\section{Introduction}
This LaTeX template provides authors with most of the formatting specifications (margins, column widths, line spacing, and text fonts) needed for preparing electronic versions of the two-page summaries. Please do not alter any of the formatting in this template when you prepare your two-page summary for the AP-S/USNC-URSI meeting.

\section{Preparing Your Paper}

The language of the symposium and the summary is English. The text should be formatted double-column, single-spaced, and sized for 215.9 $\times$ 279.4 mm (8.5 $\times$ 11 inch) paper. Left and right margins should be 15.7 mm (0.6 inches). A 25.4 mm (1 inch) margin should be present at the top and bottom of all pages. The allowed margins, and associated tolerances, for all pages are: 15.7 +/-2.5 mm (0.6 +/-0.1 in) for the left and right margins; 25.4 +/-2.5 mm (1.0 +/-0.1 in) for the top margin; and 25.4 mm (1.0 in) minimum for the bottom margin.

Summary papers are limited to two pages. Any submitted paper that exceeds this limit will be rejected.  The content of your two-page summary should substantially fill up the entire two pages (please do not make it much shorter than two pages). The page format is two-column, as illustrated here.  Do not add any kind of pagination anywhere in the paper. The font should be Times New Roman and the font size should be at least 10 point to maintain readability. All submissions will be electronic and in a PDF format.

Please take note of the following items when preparing your paper:

\subsection{Headings}
Primary section headings within the paper are enumerated by Roman numerals and are centered above the text.  Secondary section headings are enumerated by capital letters followed by periods (``A.'', ``B.'', etc.) and are flush left above their sections. The first letter of each word is capitalized. Tertiary section headings are enumerated by Arabic numerals followed by a parenthesis. They are indented, run into the text in their sections, and are followed by a colon.

\subsection{Abbreviations and Acronyms}
Define technical abbreviations and acronyms the first time they are used in the text, even if they have been defined in the abstract. Common abbreviations such as IEEE, SI, MKS, CGS, sc, dc, and rms do not have to be defined. Do not use abbreviations in the title or headings unless they are unavoidable.

\subsection{Units}
The International System of Units (SI units) is used in IEEE publications. Unit symbols should be used with measured quantities, e.g., 1 mm, but not when unit names are used in text without quantities, e.g., ``a few millimeters.'' Use a zero before decimal points: ``0.25'', not ``.25''. Include a space between the number and the unit label when used as a noun.  Do not mix complete spellings and abbreviations of units: ``Wb/m$^2$'' or ``webers per square meter,'' not ``webers/m$^2$.''  Spell out units when they appear in text: ``...a few henries,'' not ``...a few H.''

\subsection{Equations}
Equations should be labeled in consecutive numerical order. Equation numbers, within parentheses, should be aligned on the right side of the column, as in
\begin{equation}
  %\alpha+\beta=\chi.
  a+b=c.
\end{equation}         	
Define symbols before or immediately following the equation. Use ``(1)'', not ``Eq. (1)'' or ``equation (1)'', except at the beginning of a sentence: ``Equation (1) is ...''

\subsection{Grammar}

\begin{itemize}
  \item A parenthetical phrase or statement at the end of a sentence is punctuated outside of the closing parenthesis (like this). (An entire  parenthetical sentence is punctuated within the parentheses, as shown here.)
  \item There is no period after the ``et'' in the Latin abbreviation ``et al.''.
  \item The abbreviation ``i.e.'' means ``that is'', and the abbreviation ``e.g.'' means ``for example''.
  \item The subscript for the permeability of vacuum $\mu_0$, and other common scientific constants, is zero with subscript formatting, not a lowercase letter ``o.''
  
\end{itemize}

\subsection{Authors and Affiliations}

The template is designed so that author affiliations are not repeated each time for multiple authors of the same affiliation. Please keep your affiliations as succinct as possible. The authors names should appear in the order desired by the authors (the usual custom is to order the names according to the contributions of the authors, with authors who have contributed the most coming first in the list). It is recommended that emails for at least one author be included. For companies and other organizations, website addresses of the organizations  may be included, if desired.

\subsection{Figures and Tables}
Place each figure and table within the width of a single column, as in Fig.\ 1 and Table I below. Large figures and tables may span across both columns. Figure captions should be below the figures. Table headings should appear above the tables (note the capitalization). Insert figures and tables after they are cited in the text. It may be helpful to the reader if your captions, when read in succession, form a quick summary of your paper. Insert figures and tables after they are cited in the text. Use the abbreviation ``Fig.\ 1'' to reference the figure.

For axis labels, write out the quantity followed by units in parenthesis, if appropriate. For example: ``Magnetic Field (A/m).''


\begin{table}[h]
  \begin{center}
    \caption{Example of a table heading.} \label{Table1Label}
    \begin{tabularx}{0.95\columnwidth}{|X<\centering|X<\centering|X<\centering|}
      \hline
      Table Column Head 1 & Table Column Head 2 & Table Column Head 3\\
      \hline
      Item 4 & Item 5 & Item 6\\
      \hline
      Item 7 & Item 8 & Item 9\\
      \hline
    \end{tabularx}
  \end{center}
\end{table}

\begin{figure}[h]
  \begin{center}
    \noindent
    %\includegraphics[width=2.8in]{Fig1.pdf}
    \caption{The AP-S logo (top) and the UNSC-URSI logo (bottom).}\label{Fig1Label}
  \end{center}
\end{figure}


\subsection{References}

A numbered list of references must be provided at the end of the paper. The list should be arranged in the order of citation in the text, not in alphabetical order. Do not list references that are not cited in the text. Include all of the author names on a paper in the reference list; do not use ``et al.'' unless there are more than five authors. However, if this submission will be considered in the AP-S Student Paper Competition, all author names must appear for papers in the reference list. Papers that have not been published, even if they have been submitted for publication, are cited as ``unpublished'' [3]. Papers that have been accepted for publication are cited as ``in press''.

The references section head is not numbered. All references should be labeled in consecutive numerical order as they appear in the text. When citing references within the text, refer simply to the reference number enclosed by square brackets, as in [1].  Do not use ``Ref. [1]'' or ``reference [1]'' except at the beginning of a sentence: ``Reference [1] was the first...'' An ``en-dash'', not a hyphen, should be used between numbers to indicate a range of numbers, as in the beginning of this paragraph. For example, write ``pp. 1--4'', not ``pp. 1-4''.

Capitalize only the first word and proper nouns in a paper title. An example is shown in \cite{article:eason} below. The name of the journal should be in title case and in italics. For book titles, capitalize each main word of the title (i.e., use ``title case''). Book titles should be put in italics. An example is shown in [2] below. For a conference article, follow the style indicated below in \cite{conf:faraday10} and \cite{conf:faraday12}, in which the name of the conference is in title case and in italics.

The IEEE AP-S Intl. Symposium on Antennas and Propagation and the USNC-URSI Radio Science Meeting are two separate conferences, although they meet together. Please make sure that you are citing the appropriate conference.

% use section* for acknowledgement
\section*{ACKNOWLEDGEMENT}
An acknowledgement statement, if applicable, goes here before the references section. This section head is not numbered and is always singular, i.e., never ``Acknowledgements''.

% used to balance the columns on the last page adjust value as needed
%\IEEEtriggeratref{8}
% The "triggered" command can be changed if desired:
%\IEEEtriggercmd{\enlargethispage{-5in}}

% references section

% can use a bibliography generated by BibTeX as a .bbl file
% BibTeX documentation can be easily obtained at:
% http://www.ctan.org/tex-archive/biblio/bibtex/contrib/doc/
% The IEEEtran BibTeX style support page is at:
% http://www.michaelshell.org/tex/ieeetran/bibtex/
%\bibliographystyle{IEEEtran}
% argument is your BibTeX string definitions and bibliography database(s)
%\bibliography{IEEEabrv,../bib/paper}
%
% <OR> manually copy in the resultant .bbl file
% set second argument of \begin to the number of references
% (used to reserve space for the reference number labels box)
\begin{thebibliography}{1}

  \bibitem{article:eason}
  G. Eason, B. Noble, and I. N. Sneddon, ``On certain integrals of Lipschitz-Hankel type involving products of Bessel functions'', \emph{Phil. Trans. Roy. Soc. London}, vol. A247, pp. 529--551, April 1955.
  
  \bibitem{article:maxwell}
  J. Clerk Maxwell, \emph{A Treatise on Electricity and Magnetism, 3rd ed., vol. 2}. Oxford: Clarendon, 1892, pp. 68--73.
  
  \bibitem{conf:faraday10}
  M. Faraday and H. Hertz, ``Radiation characteristics of antennas,'' in \emph{Proc. Int. Symp. Antennas Propag.}, Toronto, ON, Canada, Jul. 2010, pp. 1–4.
  
  \bibitem{conf:faraday12}
  M. Faraday and H. Hertz, “More radiation characteristics of antennas,” in \emph{Proc. USNC-URSI Radio Science Meeting}, Chicago, IL, Jul. 2012, p. 1.

\end{thebibliography}

\end{document}
