\cleardoublepage

\begin{frontmatter}
\chapter[Chapter Title]{Chapter Title\footnotemark}
\footnotetext{This is chapter footnote}
\label{chap2}
\subchapter{Chapter Subtitle}

\begin{aug}
\author[addressrefs={ad1,ad2}]%
  {\fnm{Firstname}   \snm{Surname}}%
  %{\fnm{Firstname}   \snm{Surname}\footnote{This is author footnote}}%
\author[addressrefs={ad2}]%
 {\fnm{Firstname} \snm{Surname}}%
\address[id=ad1]%
  {Short Address}%
\address[id=ad2]%
  {Long Address}%
\end{aug}


% \dominitoc

%
\begin{chapterpoints}%[Chapter Points]
\item The ends of words and sentences are marked by spaces. It doesn't
  matter how many spaces you type; one is as good as 100.  The end of
  a line counts as a space.

\item The ends of words and sentences are marked by spaces. It doesn't
  matter how many spaces you type; one is as good as 100.  The end of
  a line counts as a space.
\end{chapterpoints}

\begin{dispquote}

  The ends of words and sentences are marked by spaces. It doesn't
  matter how many spaces you type; one is as good as 100.  The end of
  a line counts as a space.

  The ends of words and sentences are marked by spaces. It doesn't
  matter how many spaces you type; one is as good as 100.  The end of
  a line counts as a space.
  
  \source{Name}

\end{dispquote}

\end{frontmatter}


\section{Section title}\label{sec2.1}

Use the standard \verb|equation| environment to typeset your equations, e.g.
%
\begin{equation}
  a \times b = c\;,
\end{equation}
%
however, for multiline equations we recommend to use the \verb|eqnarray| environment\footnote{In physics texts please activate the class option \texttt{vecphys} to depict your vectors in \textbf{\itshape boldface-italic} type - as is customary for a wide range of physical subjects}.
\begin{equation} \label{eq:01}
  \begin{aligned}
    \left|\nabla U_{\alpha}^{\mu}(y)\right| &\le&\frac1{d-\alpha}\int
    \left|\nabla\frac1{|\xi-y|^{d-\alpha}}\right|\,\intd\mu(\xi) =
    \int \frac1{|\xi-y|^{d-\alpha+1}} \,\intd\mu(\xi)  \\
    &=&(d-\alpha+1) \int\limits_{d(y)}^\infty
    \frac{\mu(B(y,r))}{r^{d-\alpha+2}}\,\intd r \le (d-\alpha+1)
    \int\limits_{d(y)}^\infty \frac{r^{d-\alpha}}{r^{d-\alpha+2}}\,\intd r
  \end{aligned}
\end{equation}


\subsection*{Test reference}

Test the references: \citep{broy2022,gangolli1999,geddes1992,hamburger1995,slifka2000clinical}.

