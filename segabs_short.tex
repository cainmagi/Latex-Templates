\documentclass[wordlite,texlive,calfont,first]{segabs}

\renewcommand{\figdir}{Fig}  % figure directory

\title{Example of the shortened abstract}

\renewcommand{\thefootnote}{\fnsymbol{footnote}} 

\author{Joe Dellinger\footnotemark[1], BP 
  and Sergey Fomel, University of Texas at Austin\footnotemark[2] and Joe DellingerII\footnotemark[1], BP 
  and Sergey Fomel, University of Texas at Austin}

\footer{}
\lefthead{}
\righthead{}
\header{Joint inversion with deep learning}

\DeclareMathOperator*{\argmax}{arg\,max}
\DeclareMathOperator*{\argmin}{arg\,min}
\DeclareMathOperator*{\st}{s.t.\ }

\begin{document}

\maketitle

% While not mandatory or required, you have the opportunity to submit max two
% figures. 

\section*{Objectives and Scope}

Objectives/Scope: Please list the objectives and scope of the proposed paper. (maximum 100 words/600 characters)

Test citation~\cite[]{lamport}.

\section*{Methods}

Please briefly explain your overall approach, including your methods, procedures, and process. (maximum 250 words/1500 characters).

\section*{Results and Conclusions}

Please describe the results, observations, and conclusions of the proposed paper. (maximum 250 words/1500 characters)

\section*{Significance and Novelty}

Significance/Novelty: Please explain how this presentation will present new or additive information that can be of benefit to a practicing geoscientist. (maximum 100 words/600 characters)

\bibliographystyle{seg}  % style file is seg.bst
\autobibliography{bib/example}

\end{document}
